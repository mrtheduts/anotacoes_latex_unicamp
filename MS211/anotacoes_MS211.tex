\documentclass{article}

%encoding
%--------------------------------------
\usepackage[T1]{fontenc}
\usepackage[utf8]{inputenc}
%--------------------------------------

%Portuguese-specific commands
%--------------------------------------
\usepackage[brazil]{babel}
%--------------------------------------

%Hyphenation rules
%--------------------------------------
\usepackage{hyphenat}
\hyphenation{mate-mática recu-perar}
%--------------------------------------

%Page formatting
%--------------------------------------
\usepackage[a4paper, total={6in, 8in}, margin={1in,1in}, foot=0.5in]{geometry}
\usepackage{multicol}
\usepackage{indentfirst}
\usepackage{enumitem}
\setlist{leftmargin=1.5cm}
%--------------------------------------

%Math
%--------------------------------------
\usepackage{mathtools}
\usepackage{amsmath}
\usepackage{amssymb}
%--------------------------------------

%Plotting Graphs
%--------------------------------------
\usepackage{pgfplots}
\usepackage{tikz}
\usetikzlibrary{positioning}
\usetikzlibrary{graphs}
%--------------------------------------

%Misc
%--------------------------------------
\usepackage[bottom]{footmisc}
\usepackage{array}
\usepackage{microtype}
\usepackage{xparse}
\usepackage{booktabs}
\usepackage{float}
\usepackage{caption}
\usepackage{textcomp}
\usepackage{mdwlist}
%--------------------------------------


\frenchspacing
%\raggedbottom{}
%\raggedright{}

\begin{document}

\title{Anotações MS211\\ \large Aulas ministradas pela\\ Profª Drª Kelly Cristina Poldi}
\author{Eduardo M. F. de Souza}
\date{\today}

\pagenumbering{gobble}
\maketitle

\pagenumbering{arabic}

\tableofcontents
\newpage

\section{Erros em processos Numéricos}

    \subsection{Introdução}

        \begin{multicols}{2}

            Em diversas áreas científicas, os métodos numéricos podem ser usados para resolução de problemas. A resolução de problemas envolve várias fases:

            \begin{center}
                \begin{tikzpicture}[node distance=1.2cm]

                    \node (ProbReal) [rectangle, rounded corners, text centered, draw=black] {Problema Real};
                    \node (ModelMat) [rectangle, rounded corners, text centered, draw=black, below of=ProbReal] {Modelo Matemático};
                    \node (MetSol) [rectangle, rounded corners, text centered, draw=black, below of=ModelMat] {Método de Solução};
                    \node (AnalResul) [rectangle, rounded corners, text centered, draw=black, below of=MetSol] {Análise dos Resultados};

                    \path [->] (ProbReal) edge node [right]{Simplificações} (ModelMat);
                    \path [->] (ModelMat) edge node {} (MetSol);
                    \path [->] (MetSol) edge node {} (AnalResul);
                    \path [->] (AnalResul) edge[bend right=90] node {} (ModelMat);
                    
                \end{tikzpicture}
            \end{center}

            Uma vez resolvido um problema, pode ocorrer que a solução obtida não seja a esperada. Isto porque durante o processo de resolução pode ter ocorrido:

            \begin{enumerate}
                \item Erros de modelagem matemáticas;
                \item Erro de parâmetros;
                \item Erros associados aos sistema de numeração utilizado;
                \item Erros resultantes das operações efetuadas;
                \item Entre outros;
            \end{enumerate}

        \end{multicols}

    \subsection{Representação dos Números}

        A representação de um número depende da base escolhida (ou disponível na máquina utilizada) e o número de dígitos usados na sua representação.

        Exemplo: calcular a área de uma circunferência:
        \begin{align*}
            r &= 1000 m\\
            A &\cong 31400 m^2\\
            A &\cong 31416 m^2\\
            A &\cong 31415,92 m^2
        \end{align*}

        Quanto maior o número de dígitos utilizados, maior será a precisão obtida! Além disso, um número pode ter representação finita em uma base e não finita em outra. Por exemplo:
        \[{(0,2)}_{10} = {(0,0011\overline{0011}\ldots)}_2\]

    \subsection{Conversão de Números nos Sistemas decimal e binário}

        Sistema posicional:
        \begin{align*}
            {(6)}_{10} &= 6\times10^0\\
            {(347)}_{10} &= 3\times10^2 + 4\times10^1 + 7\times10^0\\\\
            {(10111)}_2 &= 1\times2^4 + 0\times2^3 + 1\times2^1 + 1\times2^0\\
            &= {(23)}_{10}
        \end{align*}

        De forma geral, um número $n$ na base $\beta$ é:
        \begin{gather*}
            {(a_j a{j-1} a_{j-2} \cdots a_2 a_1 a_0)}_{\beta},~\mathrm{com}~0 \leq a_k \leq \beta - 1,~\mathrm{e}~k = 1 \ldots j\\
            {(n)}_{\beta}= a_j\beta^j + a_{j-1}\beta^{j-1} + \cdots + a_2\beta^2 + a_1\beta^1 + a_0\beta^0 
        \end{gather*}

        \subsubsection{Mudança da base decimal para a base binária}

            \begin{enumerate}
                \item Parte inteira (Divisões sucessivas):
                    \begin{align*}
                        &{(6)}_{10} = {(?)}_2\\
                        6 \bmod 2 = \boxed{0} \rightarrow \quad &{(6)}_{10} = {(? \ldots \boxed{0})}_2\\
                        \lfloor  6 \div 2 \rfloor = 3 \quad &\\
                        3 \bmod 2 = \boxed{1} \rightarrow \quad &{(6)}_{10} = {(? \ldots \boxed{1}0)}_2\\
                        \lfloor  3 \div 2 \rfloor = 1 \quad &\\
                        1 \bmod 2 = \boxed{1} \rightarrow \quad &{(6)}_{10} = {(? \ldots \boxed{1}10)}_2\\
                        \lfloor  1 \div 2 \rfloor = \underline{0} \quad & \quad \textbf{(Pare!)} \\
                        &{(6)}_{10} = {(110)}_2
                    \end{align*}

                \item Parte fracionária (Multiplicações sucessivas)
                    \begin{align*}
                        &{(0,1875)}_{10} = {(?)}_2\\
                        0,1875 \times 2 = \boxed{0},375 \rightarrow \quad &{(0,1875)}_{10} = {(0.\boxed{0}\ldots~?)}_2\\
                        0,375 - \lfloor 0,375 \rfloor = 0,375 \quad &\\
                        0,375 \times 2 = \boxed{0},75 \rightarrow \quad &{(0,1875)}_{10} = {(0.0\boxed{0}\ldots~?)}_2\\
                        0,75 - \lfloor 0,75 \rfloor = 0,75 \quad &\\
                        0,75 \times 2 = \boxed{1},5 \rightarrow \quad &{(0,1875)}_{10} = {(0.00\boxed{1}\ldots~?)}_2\\
                        1,5 - \lfloor 1,5 \rfloor = 0,5 \quad &\\
                        0,5 \times 2 = \boxed{1},\underline{0} \rightarrow \quad &{(0,1875)}_{10} = {(0.001\boxed{1}\ldots~?)}_2\\
                        \textrm{\textbf{(Pare!)}} \quad &{(0,1875)}_{10} = {(0.0011)}_2
                    \end{align*}
                    
                    \paragraph{Obs:} Alguns números não têm representação finita na base binária --- um ciclo de multiplicações passa a ocorrer. O fato de um número não ter uma representação finita no sistema binário (usado nos computadores) pode acarretar na ocorrência de erros.
                    \begin{align*}
                        &{(0,1)}_{10} = {(?)}_2\\
                        0,1 \times 2 = \boxed{0},2 \rightarrow \quad &{(0,1)}_{10} = {(0.\boxed{0}\ldots~?)}_2\\
                        0,2 \times 2 = \boxed{0},4 \rightarrow \quad &{(0,1)}_{10} = {(0.0\boxed{0}\ldots~?)}_2\\
                        0,4 \times 2 = \boxed{0},8 \rightarrow \quad &{(0,1)}_{10} = {(0.00\boxed{0}\ldots~?)}_2\\
                        0,8 \times 2 = \boxed{1},6 \rightarrow \quad &{(0,1)}_{10} = {(0.0001\boxed{1}\ldots~?)}_2\\
                        0,6 \times 2 = \boxed{1},2 \rightarrow \quad &{(0,1)}_{10} = {(0.0001\boxed{1}\ldots~?)}_2\\
                        0,2 \times 2 = \boxed{0},4 \rightarrow \quad &{(0,1)}_{10} = {(0.00011\boxed{0}\ldots~?)}_2\\
                        0,4 \times 2 = \boxed{0},8 \rightarrow \quad &{(0,1)}_{10} = {(0.000110\boxed{0}\ldots~?)}_2\\
                        \vdots\\
                        &{(0,1)}_{10} = {(0,000110011\overline{0011}\ldots)}_2\\
                    \end{align*}

            \end{enumerate}


        \subsubsection{Mudança da base binária para base decimal:}

            \begin{enumerate}
                    \item Parte inteira: expoentes positivos da direita para a esquerda
                        \begin{align*}
                            {(10011)}_2 &= 1\times2^3 + 0\times2^2 + 1\times2^1 + 1\times2^0\\
                            &= 8 + 0 + 2 + 1\\
                            &= {(11)}_{10}
                        \end{align*}

                    \item Parte fracionária: expoentes negativos da esquerda para direita
                        \begin{align*}
                            {(0.101)}_2 &= 1\times2^{-1} + 0\times2^{-2} + 1\times2^{-3}\\
                            &= 0,5 + 0,125\\
                            &= {(0,625)}_{10}
                        \end{align*}
            \end{enumerate}

            \textbf{Exercício:} escreva ${(1101.011)}_2$ na base decimal:
            \begin{align*}
                {(1101.011)}_2 =&~1\times2^3 + 1\times2^2 + 0\times2^1 + 1\times2^0 +\\
                &+ 0\times2^{-1} + 1\times2^{-2} + 1\times2^{-3}\\
                =&~8 + 4 + 0 + 1 + 0 + 0,25 + 0,125\\
                =&~{(13,375)}_{10}
            \end{align*}

    \subsection{Aritmética de Ponto Flutuante}

        Um número real pode ser representado no sistema de ponto flutuante como:
        \begin{gather*}
            x = m\times\beta^e,~\textrm{em que:}\\
            m = \pm~0,d_1d_2\ldots~d_n,~n \in \mathbb{N}
        \end{gather*}

        Em que:
        \begin{itemize}
            \item $m$ é a mantissa;
            \item $n$ é o número de dígitos na mantissa;
            \item $\beta$ é a base do sistema;
            \item $e$ é o expoente, $e \in \mathbb{Z}:~l\leq~e\leq~u$, sendo que $l$ e $u$ são inteiros fixos;
            \item $d_j$ é o j-ésimo digito da mantissa, sendo que $0\leq~d_j\leq~(\beta - 1),~\textrm{com}~j = 1,~\ldots~,~n$;
            \item $d_1 \neq 0$;
        \end{itemize}

        A reunião de todos os números reais em ponto flutuante mais o zero constitui o \textbf{sistema de ponto flutuante}, denotado por $F(\beta, n, l, u)$. Nesse sistema, o menor número, em valor absoluto, é $(0,1)\times\beta^e$ e o maior valor, em valor absoluto, é $0,(\beta-1)(\beta-1)\ldots(\beta-1)\times\beta^u$. %o beta - 1 é n vezes (esses são os dígitos

        Considere o conjunto dos números reais $\mathbb{R}$ e o seguinte conjunto:
        \[G = \{x \in \mathbb{R} : {\beta}^l \leq |x| < {\beta}^u\}\]

        Dado um número real $x$, três situações podem ocorrer (Exemplos com o caso $F(10, 5, -5, 5)$):

        \begin{enumerate}
                \item $x \in G$: nesse caso, pode-se encontrar valor aproximado de $x,~\overline{x} \in F$.\\
                    Exemplo: $x = 235.89 = 0.23589 \times 10^3$
                \item $|x| < 0.1\beta^l$: $x$ está na condição de \textit{underflow}.\\
                    Exemplo: $x = 0.325 \times 10^{-7}$ ($-7 < l = -5$)
                \item $|x| \ge 0.(\beta-1)(\beta-1)~\ldots~(\beta-1) \times \beta^u$: $x$ está na condição de \textit{overflow}.\\
                    Exemplo: $x = 0.875 \times 10^{\underline{9}}$ ($9 \ge u = 5$)
        \end{enumerate}

        Exemplo: $F(\beta = 10, n = 4, l = -5, u = 5)$
        \begin{itemize}
                \item menor (abs): $0.1 \times 10^{-5}$
                \item maior (abs): $0.9999 \times 10^{5}$
        \end{itemize}
        \[G = \{x \in \mathbb{R} | 10^{-6} \leq x \leq 99990\}\]

        Outros exemplos:
        \begin{itemize}
            \item Truncamento: $x = 423.5\boxed{7} = 0.4235 \times 10^3$
            \item Arredondamento: $x = 423.5\boxed{7} = 0.4236 \times 10^3$
            \item Underflow: $x = 0.5 \times 10^{-9}~(-9 < l = -5)$
            \item Overflow: $x = 0.3 \times 10^8~(8 \ge u = 5)$
        \end{itemize}

    \subsection{Erros}

        Definimos o Erro Absoluto como ${EA}_x$ e o Erro Relativo como ${ER}_x$:
        \begin{gather*}
            {EA}_x = |x - \overline{x}|\\
            {ER}_x = \frac{{EA}_x}{|\overline{x}|} = \frac{|x - \overline{x}|}{|\overline{x}|}
        \end{gather*}

        \subsubsection{Erros de Arredondamento e Truncamento}

            Considere um sistema de ponto flutuante com $n$ dígitos e base $10$. Podemos escrever $x$ como:
            \begin{gather*}
                x = f_x \times 10^e + g_x \times 10^{e-n}~|~0.1 \leq f_x < 1,~ 0 \leq g_x < 1
            \end{gather*}

            Exemplo com $x=234,57$ e $n=4$:
            \begin{align*}
                x &= 234,57\\
                &= 2 \times 10^2 + 3 \times 10^1 + 4 \times 10^0 + 5 \times^{-1} + 7 \times 10^{-2}\\
                &= (2 \times 10^{-1} + 3 \times 10^{-2} + 4 \times 10^{-3} + 5 \times 10^{-4}) \times 10^3 + (7 \times 10^-1) \times 10^{-1}\\
                &= 0.2345 \times 10^{3} + 0.7 \times 10^{-1}\\
            \end{align*}
            
            Dessa forma, obtemos:
            \begin{align*}
                f_x &= 0.2345               &               e &= 3\\
                g_x &= 0.7 \times 10^{-1}   &               e-n &= -1\\
            \end{align*}


            Para representar $x$ nesse sistema, podemos usar dois critérios:
            \begin{itemize}
                \item \textbf{Truncamento:} $\overline{x} = f_x \times 10^e$ e $g_x \times 10^{e-n}$ é desprezado;
                    \begin{gather*}
                        |EA_x| < 10^{e-n}\\
                        |ER_x| < 10^{-n+1}
                    \end{gather*}

                    Ex: $\overline{x} = 0.2345 \times 10^3$

                \item \textbf{Arredondamento:} $\overline{x} = \begin{cases}
                        f_x  \times 10^e, & \text{se $|g_x| < \frac{1}{2}$}\\
                        f_x  \times 10^e  + 10^{e - n}, & \text{se $|g_x| \geq \frac{1}{2}$}\\
                    \end{cases}$
                    \begin{gather*}
                        |EA_x| < \frac{1}{2} \times 10^{e-n}\\
                        |ER_x| < \frac{1}{2} \times 10^{-n+1}
                    \end{gather*}

                    Ex: $\overline{x} = 0.23456 \times 10^3$
            \end{itemize}
            
            Dessa forma, temos $|EA_x|$ e $|ER_x|$ como:
            \begin{align*}
                |EA_x| &= |x - \overline{x}| \\
                &= |f_x \times 10^e + g_x \times 10^{e-n} - f_x \times 10^e|\\
                &= |g_x \times 10^{e-n}|\\
                10^{e-n} &> |g_x \times 10^{e-n}|\\
            \end{align*}

            \begin{align*}
                {ER}_x &= \frac{{EA}_x}{|\overline{x}|}\\
                &= \frac{|x - \overline{x}|}{|\overline{x}|}\\
                &= \frac{|f_x \times 10^e + g_x \times 10^{e-n} - f_x \times 10^e|}{|f_x \times 10^e|}\\
                &= \frac{g_x}{f_x} \times \frac{10^{e-n}}{10^e}\\
                &= \frac{g_x}{f_x} \times 10^{-n}\\
                \\
                \text{Como $g_x$ é dez vezes menor que}& \text{ $f_x$, temos que:}\\
                10^{-n + 1} &> \frac{g_x}{f_x} \times 10^{-n}
            \end{align*}


        \subsubsection{Teorema de Taylor}

            Suponha $f \in C^n[a,b]$, $f^{n-1}$ existe em $[a,b]$ e $x_0 \in [a,b]$
            $\forall x \in [a,b] \exists \xi(x)$ entree $x$ e $x_0$ com
            $f(x) = P_n(x) + R_n(x)$, onde:
            \[P_n(x) = f(x_0) + f^{(1)}(x_0)(x - x_0) + f^{(2)}(x_0)\frac{(x-x_0)^2}{2!} + \ldots + f^{(n)}(x_0)\frac{(x-x_0)^n}{n!}\]
            \[= \sum_{k=0}^{n} f^{(k)}(x_0) \times \frac{{(x - x_0)}^k}{k!}\]
            \[R_n(x) = f^{(n+1)}(\xi(x)) \times \frac{{(x - x_0)}^{n+1}}{(n + 1)!}\]


            $P_n(x)$ é chamado de polinômio de taylor de ordem $n$ para $f$ em torno de $x_0$ e $R_n(x)$ é o resto (erro de truncamento) associado à $P_n(x)$.

            Obs: A série infinita obtida fazendo-se o limite de $P_n(x)$ quando $n \rightarrow \inf$ é chamada de série de Taylor de $f$ em torno de $x_0$. No caso $x_0 = 0$, chamamos de polinômio de McLaurin (e série de McLaurin).

            Por exemplo, para calcular o valor de $e^{0.5}$:\\
            $f(x) = e^x$, $f^{(1)}(x) = e^x$, $f^{(2)}(x) = e^x$, \ldots~\\

            Série de Taylor ($x_0 = 0$)
            \[= f(0) + f^{(1)}(0)(x - 0) + f^{(2)}(0)\frac{(x - 0)^2}{2!} + f^{(3)}(x)\frac{(x - 0)^3}{3!} + \cdots + f^{(n)}(0)\frac{(x-0)^n}{n!}\]
            \[= e^0 + e^0(x) + \frac{e^0(x)^2}{2!} + \frac{e^0(x)^3}{3!} + \cdots + \frac{e^0x^n}{n!} + \cdots\]
            \[= 1 + x + \frac{x^2}{2!} + \frac{x^3}{3!} + \cdots + \frac{x^n}{n!} + \cdots\]

            Para o cálculo de $e^{0.5}$, precisamos \underline{truncar} a série, usando apenas um número finito de termos da série.

            Por exemplo, usando os \underline{seis primeiros termos} ($n = 5$), como aproximação:
            \[e^x \approxeq 1 + x + \frac{x^2}{2!} + \frac{x^3}{3!} + \frac{x^4}{4!} + \frac{x^5}{5!}\]
            \[e^{0.5} \approxeq 1 + 0.5 + \frac{0.5^2}{2} + \frac{0.5^3}{6} + \frac{0.5^4}{24} + \frac{0.5^5}{120}\]
            \[= 1.5 + 0.125 + 0.0208333 + 0.002604166 + 0.00026041666\]
            \[= 1.648697\]

            O erro de truncamento é dado por:
            \[R_n(x) = f^{(n=1)}(\xi_x) \times \frac{(x - x_0)^{n_1}}{(n+1)!} = f^{(6)}(\alpha) \times \frac{(x - 0)^6}{6!}\]
            Neste caso: $\frac{\alpha^6}{6!}$, com $0 \leq \alpha \leq 0.5$
            Estimativa para o erro, faço $\alpha = 0.5$
            \[R_n(x) \leq \gamma = \frac{0.5^6}{720} = 0.217 \times 10^{-6}\]

    \subsection{Zeros reais de funções reais}
        \paragraph{Definição:} um número real $\xi$ é um zero da função $f(x)$ ou uma raíz da equação $f(x)=0$ se $f(\xi)=0$.

        \begin{tikzpicture}
            \begin{axis}[
                axis lines = left,
                xlabel = $x$,
                ylabel = {$f(x)$},
            ]
            %Below the red parabola is defined
            \addlegendentry{$x^2 - 2x - 1$}
            %Here the blue parabloa is defined
            \addplot [
                domain=-10:10, 
                samples=100, 
                color=blue,
                ]
                {x^2 + 2*x + 1};

            \end{axis}
        \end{tikzpicture}

        Para calcularmos raízes reais, os métodos (numéricos) geralmente são de duas fases:

        \begin{enumerate}
            \item Determinar uma aproximação inicial: localizar ou isolar as raízes (ou seja, determinar umintervalo que contenha a raiz);
            \item Refinamento: melhorar essa aproximação usando métodos iterativos (obter aproximações dentro de uma precisão $E$ fixada);
        \end{enumerate}

        \subsubsection{Fase 1: Localizar raízes de $f(x)$}
        \paragraph{Teorema:} Seja $f(x)$ contínua em $[a,b]$. Se $f(a) \times f(b) < 0$, então pelo menos temos um ponto $\xi$ tal que $f(\xi) = 0$

        % grafico com curva decrescente que possui um \xi entre a e b

        \subsubsection*{Método gráfico}
        Pode ser utilizado para obter uma aproximação inicial para a raiz. Ela consiste em:

        \begin{enumerate}
            \item Construir o gráfico de $y=f(x)$ e obter sua intersecção com o eixo $\vec{Ox}$, \textbf{\underline{ou}}
            \item Escrever $f(x)$ na forma $g(x) = h(x)$ e obter a intersecção dos gráficos $g(x) - h(x) = 0 = f(x)$
        \end{enumerate}

        \paragraph{Exemplo:}
        \begin{enumerate}[label=\alph*]
            \item \begin{multicols}{2}
                $f(x) = x^2 - x - 2 = 0$

                \begin{tikzpicture}
                    \begin{axis}[
                        axis lines = center,
                        xlabel = $x$,
                        ylabel = {$f(x)$},
                    ]
                    %Below the red parabola is defined
                    %Here the blue parabloa is defined
                    \addplot [
                        domain=-2:3, 
                        samples=100, 
                        color=blue,
                        ]
                        {x^2 - x - 2};
                        \addlegendentry{$f(x)$}
                    \end{axis}
                \end{tikzpicture}

                \vfill

                \begin{gather*}
                    g(x) = h(x)\\
                    x^2 = x + 2
                \end{gather*}

                \begin{tikzpicture}
                    \begin{axis}[
                        axis lines = center,
                        xlabel = $x$,
                        ylabel = {$f(x)$},
                    ]
                    %Below the red parabola is defined
                    %Here the blue parabloa is defined
                    \addplot [
                        domain=-2:3, 
                        samples=100, 
                        color=blue,
                        ]
                        {x^2};
                        \addlegendentry{$g(x)$}
                    \addplot [
                        domain=-2:3, 
                        samples=100, 
                        color=red,
                        ]
                        {x + 2};
                        \addlegendentry{$h(x)$}
                    \end{axis}
                \end{tikzpicture}

        \end{multicols}
        \item $f(x) = x^2 - e^x = 0$

                \begin{gather*}
                    g(x) = h(x)\\
                    x^2 = x + 2
                \end{gather*}

            \item $f(x) = \log{x} + x = 0$ (Tarefa!)

        \end{enumerate}

        \subsubsection*{Fase 2: Refinamento}
        Métodos iterativos para se obter zeros de funções:
        \begin{itemize}
            \item ponto inicial;
            \item sequência de instruções (iterações);
            \item critério de parada;
        \end{itemize}

        \begin{gather*}
            |-f(x)| < E\\
            |x_{k-1} - x_k| < E
        \end{gather*}

        \subsubsection*{Método da Bisseção}
        Seja $f$ uma função contínua em $[a,b]$ e suponha $f(a) \times f(b) < 0$ (ou seja, existe raiz real).
        Para simplificar, vamos supor que existe uma única raíz real em $[a,b]$

        O método da bisseção consistem em determinar uma sequência de intervalos $[a_i,b_i],~i = 1,~2,~\ldots$ de tal forma que $[a_i,b_i]$ sempre contenha a raíz.

        % Grafico que tem uma função que passa do negativo para o positivo no intervalo [a,b], sendo que x_1 = (a1+b1)/2. Iterativamente, vai diminuindo a e b pelo novo valor de x

        De forma geral:
        \begin{gather*}
            x_i = \frac{a_i + b_i}{2}\\
            f(a_i) \times f(x_i) =
                \begin{cases}
                    < 0,~ & \text{então}
                        \begin{cases}
                            a_{i+1} = a_i\\
                            b_{i+1} = x_i\\
                        \end{cases}\\
                    > 0,~ & \text{então}
                        \begin{cases}
                            a_{i+1} = x_i\\
                            b_{i+1} = b_i\\
                        \end{cases}\\
                \end{cases}\\
        \end{gather*}

        \textbf{Critério de Parada:}
        \[|b_k - a_k| < E\]

        \subsubsection*{Estimativa para o número de iterações}
        Na iteração $k$, temos que:
        \begin{align*}
            b_k - a_k &= \frac{b_{k-1} - a_{k-1}}{2}\\
            &= \frac{b_0 - a_0}{2^k}
        \end{align*}

        Queremos saber o valor de $k$ tal que $B_k - a_k < E$, ou seja:

        \begin{gather*}
            \frac{b_0 - a_0}{2^k} < E\\
            \rightarrow~2^k E > b_0 - a_0\\
            \rightarrow~2^k > \frac{b_0 - a_0}{E}\\
            \text{Aplicando log em ambos os lados}\\
            k \times \log{2} > \log{(b_0 - a_0)} - \log{E}\\
            \boxed{k > \frac{\log{(b_0 - a_0)} - \log{E}}{\log{2}}}
        \end{gather*}

        \paragraph{Exemplo:} $f(x) = x^3 + 3x - 1= 0$, com $[a,b] = [0,1]$ e $E = 10^{-1} = 0,1$
        \begin{gather*}
            f(a = a_1) = f(0) = -1 < 0~\text{e}~f(b = b_1) = f(1) = 3 > 0\\
            f(a) \times f(b) = 0 \rightarrow \text{Ok!}\\
            k = 1\\
            x_1 = \frac{a_1 + b_1}{2} = \frac{0 + 1}{2} = 0,5\\
            \rightarrow f(b_2 = 0,5) = 0,625 > 0\\
            \xi \in [0,0.5] \rightarrow |b_2 - a_2| = 0,5 > E\\
            k = 2\\
            x_2 = \frac{a_2 + b_2}{2} = \frac{0 + 0,5}{2} = 0,25\\
            \rightarrow f(a_3 = 0,25) = -0,23437 < 0\\
            \xi \in [0.25,0.5] \rightarrow |b_3 - a_3| = 0,5 > E\\
            k = 3\\
            x_3 = \frac{a_3 + b_3}{2} = \frac{0,25 + 0,5}{2} = 0,375\\
            \rightarrow f(b_4 = 0,375) = 0,1777 > 0\\
            \xi \in [0.25,0.375] \rightarrow |b_2 - a_2| < 0,125 > E\\
            k = 4\\
            x_4 = \frac{a_4 + b_4}{2} = \frac{0,25 + 0,375}{2} = 0,3125\\
            \rightarrow f(a_5 = 0,3125) = -0,0319824 < 0\\
            \xi \in [0.3125,0.375] \rightarrow |b_5 - a_5| < 0,0625 < E~\textbf{(Pare!)}\\
        \end{gather*}

        Estimativa para o número de iterações:
        \begin{cases*}
            a_0 = 0\\
            b_0 = 1\\
        \end{cases*}, $E = 0,1$.

        \begin{gather*}
            k > \frac{\log{(b_0 - a_0)} - \log{E}}{\log{2}}\\
            k > \frac{\log{(1 - 0)} - \log{0.1}}{\log{2}}\\
        \end{gather*}

\end{document}
